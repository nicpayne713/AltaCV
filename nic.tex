%%%%%%%%%%%%%%%%%
% This is an sample CV template created using altacv.cls
% (v1.6.4, 13 Nov 2021) written by LianTze Lim (liantze@gmail.com). Now compiles with pdfLaTeX, XeLaTeX and LuaLaTeX.
%
%% It may be distributed and/or modified under the
%% conditions of the LaTeX Project Public License, either version 1.3
%% of this license or (at your option) any later version.
%% The latest version of this license is in
%%    http://www.latex-project.org/lppl.txt
%% and version 1.3 or later is part of all distributions of LaTeX
%% version 2003/12/01 or later.
%%%%%%%%%%%%%%%%

%% Use the "normalphoto" option if you want a normal photo instead of cropped to a circle
% \documentclass[10pt,a4paper,normalphoto]{altacv}

\documentclass[10pt,a4paper,ragged2e,withhyper]{altacv}
%% AltaCV uses the fontawesome5 and packages.
%% See http://texdoc.net/pkg/fontawesome5 for full list of symbols.

% Change the page layout if you need to
\geometry{left=1.25cm,right=1.25cm,top=0.5cm,bottom=1.5cm,columnsep=1.2cm}

% The paracol package lets you typeset columns of text in parallel
\usepackage{paracol}

% Change the font if you want to, depending on whether
% you're using pdflatex or xelatex/lualatex
\ifxetexorluatex
  % If using xelatex or lualatex:
  \setmainfont{Roboto Slab}
  \setsansfont{Lato}
  \renewcommand{\familydefault}{\sfdefault}
\else
  % If using pdflatex:
  \usepackage[rm]{roboto}
  \usepackage[defaultsans]{lato}
  % \usepackage{sourcesanspro}
  \renewcommand{\familydefault}{\sfdefault}
\fi

% Change the colours if you want to
% \definecolor{SlateGrey}{HTML}{2E2E2E}
% \definecolor{LightGrey}{HTML}{666666}
% \definecolor{DarkPastelRed}{HTML}{450808}
% \definecolor{PastelRed}{HTML}{8F0D0D}
% \definecolor{GoldenEarth}{HTML}{E7D192}
% \colorlet{name}{black}
% \colorlet{tagline}{PastelRed}
% \colorlet{heading}{DarkPastelRed}
% \colorlet{headingrule}{GoldenEarth}
% \colorlet{subheading}{PastelRed}
% \colorlet{accent}{PastelRed}
% \colorlet{emphasis}{SlateGrey}
% \colorlet{body}{LightGrey}

\definecolor{VividPurple}{HTML}{008080}
\definecolor{SlateGrey}{HTML}{2E2E2E}
\definecolor{LightGrey}{HTML}{666666}
\colorlet{heading}{VividPurple}
\colorlet{accent}{VividPurple}
\colorlet{emphasis}{SlateGrey}
\colorlet{body}{LightGrey}
% Change some fonts, if necessary
\renewcommand{\namefont}{\Huge\rmfamily\bfseries}
\renewcommand{\personalinfofont}{\footnotesize}
\renewcommand{\cvsectionfont}{\LARGE\rmfamily\bfseries}
\renewcommand{\cvsubsectionfont}{\large\bfseries}


% Change the bullets for itemize and rating marker
% for \cvskill if you want to
\renewcommand{\itemmarker}{{\small\textbullet}}
\renewcommand{\ratingmarker}{\faCircle}

%% Use (and optionally edit if necessary) this .cfg if you
%% want to use an author-year reference style like APA(6)
%% for your publication list
\input{pubs-authoryear.cfg}

%% Use (and optionally edit if necessary) this .cfg if you
%% want an originally numerical reference style like IEEE
%% for your publication list
% \input{pubs-num.cfg}

%% sample.bib contains your publications
\addbibresource{sample.bib}

\begin{document}
\name{Nicholas Payne}
\tagline{Data Engineering & Solutions Architecture}
%% You can add multiple photos on the left or right
% \photoR{3.5cm}{bitmoji}
\photoR{3.5cm}{me}
% \photoL{2.5cm}{Yacht_High,Suitcase_High}

\personalinfo{%
  % Not all of these are required!
  \email{nic.payne@protonmail.com}
  % \phone{319-389-5740}
  \location{WI}
  \twitter{@pypeaday}
  % \linkedin{nicpayne713}
  \github{pypeaday}
  \printinfo{\faTwitch}{pypeaday}[https://twitch.tv/pypeaday]
  %% You can add your own arbitrary detail with
  %% \printinfo{symbol}{detail}[optional hyperlink prefix]
  % \printinfo{\faPaw}{Hey ho!}[https://example.com/]
  %% Or you can declare your own field with
  %% \NewInfoFiled{fieldname}{symbol}[optional hyperlink prefix] and use it:
  % \NewInfoField{gitlab}{\faGitlab}[https://gitlab.com/]
  % \gitlab{your_id}
  %%
  %% For services and platforms like Mastodon where there isn't a
  %% straightforward relation between the user ID/nickname and the hyperlink,
  %% you can use \printinfo directly e.g.
  % \printinfo{\faMastodon}{@username@instace}[https://instance.url/@username]
  %% But if you absolutely want to create new dedicated info fields for
  %% such platforms, then use \NewInfoField* with a star:
  % \NewInfoField*{mastodon}{\faMastodon}
  %% then you can use \mastodon, with TWO arguments where the 2nd argument is
  %% the full hyperlink.
  % \mastodon{@username@instance}{https://instance.url/@username}
}

\makecvheader
%% Depending on your tastes, you may want to make fonts of itemize environments slightly smaller
% \AtBeginEnvironment{itemize}{\small}

%% Set the left/right column width ratio to 6:4.
\columnratio{0.6}

% Start a 2-column paracol. Both the left and right columns will automatically
% break across pages if things get too long.
\begin{paracol}{2}
\cvsection{Caterpillar, Inc.}

\cvevent{Data Engineering, Solutions Architecture, DevOps}{Group: Remanufacturing}{February 2021-Present}{Remote}


\begin{itemize}

\item Design ETL pipeline solutions with Python and Kedro targeting Docker deployment on AWS Batch with Snowflake and S3 data stores
\item Manage AWS infrastructure including S3, Batch, ECS, Sagemaker, CloudWatch, and EventBridge with CloudFormation templates
\item Coach data scientists on my team in clean coding, best practices such as git, advancted python coding for data engineering, and Kedro pipeline deployment
\item Manage Azure Build Pipelines and help enable/mature our CI/CD strategy
\item Build Streamlit apps for visualizing and exploring data with customers to guide our Kedro pipeline developments
\item Authored utility library, `rada`, which enables data scientists to quickly iterate on features locally while deploying their code into AWS Batch without waiting on a CI pipeline, right from their terminal

\end{itemize}

\divider

\cvevent{Data Engineer/Scientist}{Groups: Excavation Division {\faDAndD} Integrated Components and Solutions}{January 2019-February 2021}{Mossville, IL}
\begin{itemize}
\item  Co-authored Python library for interfacing with controls software of Caterpillar Excavators for automated data gathering and ML activities
\item  Dockerized ML Pipeline deployment using docker-compose, Python, and python-rq
\item  Designed an automated data ingestion and grooming pipeline to index and organize data which included deep learning object detection modeling for tagging datasets with desired information
\item  Designed an Exploratory Data Analysis (EDA) web application with streamlit to explore the data ingested with above-mentioned data pipeline
\item  Data modeling for commonizing data storage pracitces in the perception space at Caterpillar
\item  Built a data storage prototype utilizing MySQL and Flask to centralize telematics and kinematics data
\end{itemize}

\divider

\cvevent{Data Scientist}{Group: Information Analytics}{July 2017-January 2019}{Peoria, IL}
\begin{itemize}
\item  Graduated from the Analytics Professional Development Program (2019)
\item  Worked on Scrum team following Agile development
\item  Gained experience in building machine learning models using various frameworks in Python for random forest regressions models, reinforcement learning models, and various deep learning models for computer vision applications
\end{itemize}

% \cvevent{Head of Staff}{Summit Ministries}{Summers of 2015-2017}{Manitou Springs, CO}
% \begin{itemize}
% \item Oversaw staff of 35-45 camp counselors and ~180 students for 2-week conferences during the summers
% \item Coached counselors in discussion facilitation, job duties, and techniques for handling sensitive issues such as student misconduct
% \item Led efforts to find solutions for situations with disgruntled students and/or parents
% \item Gained valuable leadership skills in the areas of personal connection, adaptation, critical thinking, and situation analysis

% \end{itemize}

\cvsection{Related Experience}

\cvevent{Hobby Homelabbing}{My house}{\faInfinity}{\faHome}

\begin{itemize}

\item Utilize Ansible for deploying ~40 containerized workloads to a self-hosted Ubuntu server
\item Manage Jellyfin and Nextcloud instances with multiple users
\item Self-host Docker container registry for keeping my used images up to date
\item I gain regular experience in systematic thinking in a technical context of modern tooling which as added value to every one of my roles and projects

\end{itemize}

\divider

\cvsection{Certificates}

\begin{itemize}
\item Fundamentals of Deep Learning for Computer Vision | \textit{Nvidia}
\item Convolutional Neural Networks | \textit{Coursera}
\item Structuring Machine Learning Projects | \textit{Coursera}
\item Improving Deep Neural Networks: Hyperparameter tuning, Regularization and Optimization | \textit{Coursera}
\item Neural Networks and Deep Learning | \textit{Coursera}
% \item Python Programmer | \textit{Data Camp}
% \item Data Scientist with Python | \textit{Data Camp}
\end{itemize}


\cvsection{Hobbies}

\cvachievement{\faGlassMartini}{Whisky tasting and cigar pairing}{Enjoying a nice dram}
%\divider
\cvachievement{\faServer}{Home-labbing}{Self host all the things!}

\cvachievement{\faSpinner}{Theology}{Ancient Near Eastern Cosmology as the roots for Biblical exegesis}
%\divider
% \cvachievement{\faVolleyballBall}{Volleyball}{Because ball is life}
\cvachievement{ \faHandPeace}{BJJ}{6 year white belt}
%\divider
% \cvachievement{\faRocket}{Travelling}{From the beaches of Barbados to the green hills of Ireland}
%\divider
\cvachievement{\faJoget}{Disc Golf}{Olympics 2024. You heard it here first}

%% Switch to the right column. This will now automatically move to the second
%% page if the content is too long.
\switchcolumn

\cvsection{Looking for}

\begin{quote}
\justify
% A team of other disciplined students and tenacious learners who look for ways to improve their lives through automation and love working with data to make data-driven decisions.
A team of other disciplined students and tenacious learners who continuously look for ways to improve their live by striving for excellence in their respective passions.
\end{quote}

\cvsection{Technical Skills}

\cvevent{{\faPython} Python }{}{}{}

\begin{itemize}
\item \textit{{\faStream} Pipeline} | Kedro
\item \textit{{\faAtom} Data Engineering/Science} | Pandas, Numpy, Scipy, Sqlalchemy
% \item \textit{{\faMicrochip} Computer Science} | Multiprocessing, Python-rq
\item \textit{{\faEye} Exploration} | Streamlit, Jupyter Notebooks
\item \textit{{\faChartPie} Viz} | Plotly, Matplotlib, Seaborn, Hvplot
\item \textit{{\faCodeBranch} Versioning} | MLflow, \faGit

\item \textit{{\faProjectDiagram} Machine Learning / Deep Learning Frameworks} | Keras / Tensorflow, PyTorch with APex and AMP, SciKit-Learn
\item \textit{{\faBrain} Machine Learning / Deep Learning Algorithms} | Standard sk-learn techniques, object detection + classification

\end{itemize}

\cvevent{{\faAws} AWS }{}{}{}

\begin{itemize}

\item \textit{{\faStream} Services} | Batch, EC2, S3, EventBridge, CI/CD, boto3

\end{itemize}

\cvevent{{\faDatabase} SQL }{}{}{}
\begin{itemize}
\item Schema design and data modeling
\item \textit{Databases} | MySQL, Oracle, MS SQL Server, Snowflake
\end{itemize}

\cvevent{{\faDocker} Containerization }{}{}{}

\begin{itemize}
\item Docker(-Compose)
\end{itemize}

\cvevent{{\faChartArea} Visualization }{}{}{}
\begin{itemize}
\item Experience building EDA tools with \textit{streamlit}
\end{itemize}

\cvevent{ {\faServer} Miscellaneous}{}{}{}
\begin{itemize}
\item {\faLinux} Linux
\item Remote deployment/development
\item Vim is the superior editor
\item \LaTeX
\item  {\faGit}
\end{itemize}

\divider

%% Add a cheeky quote
\vspace*{\fill}
\hspace*{\fill}
q: Which came first, the phoenix or the egg?

\newpage

\cvsection{Strengths}
\cvtag{Active Learning}
\cvtag{Leadership}
\cvtag{Coaching}
\cvtag{Flexibility and Adaptability}
\cvtag{Conflict Resolution}
\cvtag{Analytical Thinking}
%% Yeah I didn't spend too much time making all the
%% spacing consistent... sorry. Use \smallskip, \medskip,
%% \bigskip, \vspace etc to make adjustments.
\medskip

\cvsection{Education}
\cvevent{MS Applied Mathematics} {Iowa State University}{2014 - 2016}{Ames, IA}
\textsc{CGPA}: 3.84/4.0
\\
\cvevent{BS Mathematics}{Iowa State University}{2011-2014}{Ames, IA}
\textsc{CGPA}: 3.64/4.0


\cvsection{Patents}

\begin{itemize}
\item \textbf{2022} | Excavator Control Mapping System
\item \textbf{filed} | Vision Based Object Detection Alarm Snooze Strategy for Rotating Machines

\end{itemize}


\cvsection{Conferences}

\begin{itemize}
\item neovim.conf 2022 (speaker) | 2021 (participant)
\item Nvidia GPU Technology Conference 2019 \& 2020
\item Hackillinois 2018

\end{itemize}

\cvsection{Publications}

% \printbibliography[heading=pubtype,title={\printinfo{\faFile*[regular]}{Journal Articles}},type=article]

\begin{itemize}
\item {\faFile*[regular]} \textbf{Properties Preserving Schemes for a Kinetic Eikonal Equation} | \textit{J. Comput. Phys. 331(2016)}
\item {\faFile*[regular]} \textbf{An asymptotic method based on a Hopf-Cole transformation for a kinetic BGK equation in the hyperbolic limit} | \textit{J. Comput. Phys. 341: 295-312 (2017)}
\item {\faFile*[regular]} \textbf{A Hopf-Cole transformation based asymptotic method for kinetic equations with a BGK collision operator in the large scale hyperbolic limit} | \textit{Iowa State University Graduate Theses and Dissertations. 15788.}

\end{itemize}



\cvsection{References}


Available on request




\end{paracol}


%% Add a cheeky quote
\vspace*{\fill}
\hspace*{\fill}
% All models are wrong, some models are useful
a: A circle has no beginning

\clearpage

\end{document}
\end{document}
